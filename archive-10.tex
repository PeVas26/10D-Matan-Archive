\documentclass{book}
\usepackage[utf8]{inputenc}
\usepackage[warn]{mathtext}
\usepackage[T2A]{fontenc}
\usepackage[utf8]{inputenc}
\usepackage[russian]{babel}
\usepackage{amsmath}
\usepackage{amssymb}
\usepackage[left=2cm,right=2cm,
    top=2cm,bottom=2cm,bindingoffset=0cm]{geometry}
\begin{document}

\section*{25. Бесконечно Большие Последовательности}

\paragraph{25.15а.}
\textit{Докажите, что у всякой последовательности длины $n^2 + 1$ существует монотонная подпоследовательность длины $n+1$.}

Пусть максимальная монотонная последовательность состоит из n членов. Возьмем для каждого числа в последовательности пару чисел — длин максимальных убывающей и возрастающей монотонных последовательностей, заканчивающихся на этом элементе. Всего таких пар не более $n^2$. Докажем, что 2 одинаковых пар нет. Пусть они есть у членов i-ого и j-ого. Тогда если i>j, то i-ый элемент будет продолжением либо возрастающей, либо убывающей монотонной последовательности, заканчивающейся на j-том, тогда пары разные. Тогда по принципу Дирихле 2 пары одинаковые, чего быть не может, а значит максимальная длина — n+1.

\paragraph{25.15б.}
\textit{Останется ли верным утверждение задачи в случае последовательности длины $n^2$ при больших значениях $n$?}

Отсортируем последовательность длины $n^2$, состоящую из разных по значению членов, в порядке возрастания. Разделим ее на куски длины n и сделаем новую последовательность, поставив куски в обратном порядке. Например, если n = 3, последовательность будет такой: , 2 вертикальные черты разделяют куски. Теперь о том, почему тут нет монотонной последовательности длины n + 1. Рассмотрим возрастающие монотонные подпоследовательности. Рассмотрим какой-то член. Тогда после него больше него будут члены только в его куске, то есть максимум длины такой подпоследовательности n. Рассмотрим убывающие монотонные последовательности. Т.к. в каждом куске члены возрастают, то не будет подпоследовательности, в которой будет более 1 члена из какого-то куска, то есть длина тоже будет n. Тогда получаем, что такого n нет.

\medskip\hrule\medskip

\section*{26. Бесконечно Малые Последовательности}
 
\paragraph{26.1a.}
\textit{Запишите в кванторах определение бесконечно малой последовательности. Запишите их так, чтобы кванторы
существования шли в начале формулы.}

$\exists f: \mathbb{R} \to \mathbb{N}$ $\forall \varepsilon > 0 \in \mathbb{R}$, $k \in \mathbb{N}$ $k>f(\varepsilon) \rightarrow |a_k| < \varepsilon$

\paragraph{26.1б.}
\textit{Запишите в кванторах определение последовательности, не являющейся бесконечно малой. Запишите их так, чтобы кванторы
существования шли в начале формулы.}

$\exists \varepsilon > 0$ $\exists f: \mathbb{N} \to \mathbb{N}$ $\forall N \in \mathbb{N}$ $|a_{f(N)}| > \varepsilon$ или $f(N) > N$

\paragraph{26.2а.}
\textit{Для последовательности $\{\alpha_n\}$ укажите какой-нибудь номер N, начиная с которого для всех членов последовательности верно неравенство $|\alpha_n| < \varepsilon$, если $\alpha_n = \frac{1}{n}$}

$\frac{2}{\varepsilon}$. Тогда при таком n $a_n = \frac{\varepsilon}{2} < \varepsilon$.

\paragraph{26.2б.}
\textit{Для последовательности $\{\alpha_n\}$ укажите какой-нибудь номер N, начиная с которого для всех членов последовательности верно неравенство $|\alpha_n| < \varepsilon$, если $\alpha_n = \frac{(-1)^n \cos n}{n^3}$}

Т.к. нужен модуль, степень -1 нас не волнует. Т.е. выражение превращается в $\frac{\cos n}{n^3} < \frac{\cos n}{n}$. Т.к. cos n < 1, то при $n=\frac{1}{\varepsilon} $ получаем $\varepsilon \cos \varepsilon < \varepsilon$ 

\paragraph{26.2в.}
\textit{Для последовательности $\{\alpha_n\}$ укажите какой-нибудь номер N, начиная с которого для всех членов последовательности верно неравенство $|\alpha_n| < \varepsilon$, если $\alpha_n = 0.99^n$}

При $n = log_{0.99} \varepsilon$ $a_n = 0.99^{2log_{0.99}\varepsilon} = \varepsilon^2 < \varepsilon$ при $\varepsilon < 1$.

\paragraph{26.2г.}
\textit{Для последовательности $\{\alpha_n\}$ укажите какой-нибудь номер N, начиная с которого для всех членов последовательности верно неравенство $|\alpha_n| < \varepsilon$, если $\alpha_n = \frac{2n}{n!}$}

После n = 20 $\frac{2n}{n!} $ изменяется за одно n более, чем на порядок, из-за умножения на числа, меньшие $\frac{2}{20}$ на каждом шаге. Тогда достаточное n равно кол-ву первых 0 в десятичной записи $\varepsilon$ + 20. 

\paragraph{26.2д.}
\textit{Для последовательности $\{\alpha_n\}$ укажите какой-нибудь номер N, начиная с которого для всех членов последовательности верно неравенство $|\alpha_n| < \varepsilon$, если $\alpha_n = \frac{2n + 3}{n^2 + 2n - 1}$}

При натуральных n $\frac{1}{n^2+2n-1} < \frac{1}{n^2}$; $2n + 3 < 3n$. Значит, $\frac{2n+3}{n^2+2n-1} < \frac{3n}{n^2} = \frac{3}{n}$, которое меньше $\varepsilon$ при $n = \frac{1}{1.5 \varepsilon}$
\newpage

\paragraph{26.2е.}
\textit{Для последовательности $\{\alpha_n\}$ укажите какой-нибудь номер N, начиная с которого для всех членов последовательности верно неравенство $|\alpha_n| < \varepsilon$, если $\alpha_n = \sqrt{n + 1} - \sqrt{n}$}

\textit{Для последовательности $\{\alpha_n\}$ укажите какой-нибудь номер N, начиная с которого для всех членов последовательности верно неравенство $|\alpha_n| < \varepsilon$, если $a_n = \sqrt{n+1} - \sqrt{n}$}.

\[\sqrt{n+1} - \sqrt{n} < \varepsilon\]
\[\sqrt{n+1} < \varepsilon + \sqrt{n}\]
\[n+1 < \varepsilon^2 + n + 2\sqrt{n}\varepsilon\]
\[\varepsilon^2 + 2 \varepsilon \sqrt{n} - 1 > 0\]
\[\frac{1 - \varepsilon^2}{2 \varepsilon}\]
\[\sqrt{n} > \frac{1 - \varepsilon^2}{2 \varepsilon}\]
\[n > \frac{(1 - \varepsilon^2)^2}{{4\varepsilon^2}} \]

\paragraph{26.3а.}
\textit{Верно ли, что сумма БМП также является БМП?}

Да. Пусть есть сумма 2 бесконечно малых последовательностей (БМП). Тогда для каждого $\varepsilon$ по определению БМП найдется такие $N_1$ и $N_2$, что для каждого $n_1 > N_1$ и $n_2 > N_2$ верно $|a_n|<\frac{\varepsilon}{2}$ и $|b_n|<\frac{\varepsilon}{2}$. Тогда $|a_n + b_n| < \varepsilon$, для всех n, больших большего из $N_1$ и $N_2 $, что является определением БМП.

\paragraph{26.3б.}
\textit{Верно ли, что разность БМП также является БМП?}

Если последовательность БМП, то и обратная ей БМП, т.к.$|x| = |-x|$ . Разность 2 БМП — сумма 1-й БМП и обратной 2-й БМП, т.е. сумма БМП, то есть верно.

\paragraph{26.3в.}
\textit{Верно ли, что произведение БМП также является БМП?}

У каждой БМП есть N, после которого модуль всех членов меньше 1. Поэтому для них если $|a_n|<\varepsilon$; $|b_n|<\varepsilon$, то $|a_n \cdot b_n|<|a_n|$ и $|a_n \cdot b_n| < b_n$ для всех n, больших большего из $N_1$ и . 

\paragraph{26.3г.}
\textit{Верно ли, что отношение БМП также является БМП?}

Нет, пример $a_n = \frac{1}{n}$; $b_n =\frac{1}{n^2} \rightarrow \frac{a_n}{b_n} = n$, что не БМП.

\paragraph{26.4.}
\textit{Пусть последовательности $\{\alpha_n\}$ и $\{\beta_n\}$ являются БМП, а последовательность $\{\gamma_n\}$ такова, что $\alpha_n \leq \gamma_n \leq \beta_n$. Докажите, что тогда последовательность $\{\gamma_n\}$ также является БМП.}

Рассмотрим какой-нибудь $\varepsilon$. Тогда в БМП $\alpha$ и $\beta$ есть $N_1$ и $N_2$ такие, что для всех n таких, что $n>N_1$ и $n>N_2$, верно $b_n \le a_n $. Тогда тоже БМП.

\paragraph{26.5}
\textit{Верно ли, что последовательность $\{\alpha_n\}$ с отличными от нуля членами является БМП тогда и только тогда, когда последовательность $\{\frac{1}{\alpha_n}\}$ является ББП?}

Будем брать $C = \frac{1}{\varepsilon}$. Тогда есть однозначное соответствие между $\varepsilon$ и С. Возьмем бесконечно большую последовательность. Тогда, если в ней нет нулей, то есть последовательность $\beta_n = \frac{1}{\alpha_n}$ ,где $\alpha_n$ - бесконечно большая последовательность. Тогда $\beta$ очевидно бесконечно малая, и очевидно, что каждой ББП минимум 1 БМП. Возьмем БМП без нулей. Проделав аналогичную операцию, получим ББП. Т.е. есть биекция между БМП и ББП. Это перефразированный вопрос задачи.

\paragraph{26.6a.}
\textit{Верно ли, что последовательность $\{\alpha_n\}$ с  положительными членами является БМП тогда и только тогда, когда последовательность $\{\alpha_n^2\}$ является БМП?}

Если $\alpha$ - БМП, то $\beta$ такая, что $b_n = |a_n|$, тоже, очевидно, БМП. У такой БМП найдется N-ый член, после которого все члены БМП меньше 1. Расставим знаки сравнения между членами. Тогда в последовательности $\{a_n^2\}$ после N-ого члена будет такой же порядок следования знаков, при этом каждый член будет по модулю меньше соответствующего в изначальной. Тогда это БМП. Аналогичные действия проходят и в обратную сторону. 

\paragraph{26.6б.}
\textit{Верно ли, что последовательность $\{\alpha_n\}$ с  положительными членами является БМП тогда и только тогда, когда последовательность $\{\sqrt[3]{\alpha_n}\}$ является БМП?}

Если $\alpha$ - БМП, то $\beta$ такая, что $b_n = |a_n|$, тоже, очевидно, БМП. У такой БМП найдется N-ый член, после которого все члены БМП меньше 1. Расставим знаки сравнения между членами. Тогда в последовательности $\{\sqrt[3]{a_n}\}$ после N-ого члена будет такой же порядок следования знаков, при этом каждый член будет по модулю больше соответствующего в изначальной, но меньше 1. Тогда это БМП. Аналогичные действия проходят и в обратную сторону. 

\paragraph{26.7.}
\textit{Последовательности $\{\alpha_n\}$ и $\{\beta_n\}$ БМП, а последовательность $\{\gamma_n\}$ такова, что $gamma_{2n -1} = \alpha_n, \gamma_{2n} = \beta_n$ при любом натуральном n. Является ли последовательность $\{\gamma_n\}$ БМП?}

Да, является. Возьмем $\varepsilon$. Тогда начиная с членов $N_1$ и $N_2$ все последующие будут меньше. Возьмем большее из них. Тогда, домножив его на 2, получим номер члена в $\gamma$, начиная с которого все остальные меньше $\varepsilon$.

\paragraph{26.8a.}
\textit{Одна последовательность БМП, а другая ограниченная. Что можно сказать об их сумме?}

БМП ограниченна, поэтому, т.к. сумма 2 ограниченных ограниченна, то и полученная будет ограничена. Однако это не БМП. Пример — ограниченная 1,1,1,1… и БМП $1, \frac{1}{2}, \frac{1}{3}, \frac{1}{4}$, ... . Ее члены всегда больше 1, что не подходит под определение БМП.

\paragraph{26.8б.}
\textit{Одна последовательность БМП, а другая ограниченная. Что можно сказать об их произведении?}

Получится БМП. Возьмем большее по модулю ограничение огранченной - k. Тогда ее члены все меньше k по модулю. Теперь возьмем $\varepsilon$. Результат будет БМП, если для любого $\varepsilon$ в БМП найдется такой член, после которого все члены будут меньше $\frac{\varepsilon}{k}$ . Это число положительное, поэтому по определению БМП такой член найдется. Поэтому результат БМП.

\paragraph{26.8в.}
\textit{Одна последовательность БМП, а другая ограниченная. Что можно сказать об их отношении?}

Если в последовательностях есть нули, то результат, очевидно, не определен. Если же их нет, то результат может ББП (пример в 26.8а), БМП (перевернуть отношение из 26.8а), ограниченной (2 одинаковые БМП), неограниченной (последовательности $1. \frac{1}{2}, 1, \frac{1}{3}, 1, \frac{1}{4}, 1, ...$ и $1, \frac{1}{2}, \frac{1}{3}, \frac{1}{4}, ...$).

\paragraph{26.9а.}
\textit{Одна последовательность БМП, а другая ББП. Что можно сказать об их сумме?}

Получится ББП, т.к. это сумма ББП и ограниченной, которая ББП.

\paragraph{26.9б.}
\textit{Одна последовательность БМП, а другая ББП. Что можно сказать об их произведении?}

Может получиться ББП ($\{n^2\}$ и $\{{n^{-1}}\}$), БМП ($\{n^2\}$ и $\{n\}$), ограниченной ($\{n\}$ и $\{n^{-1}\}$), неограниченной (первый пример, если в БМП каждый второй член заменить на 0).

\paragraph{26.9в.}
\textit{Одна последовательность БМП, а другая ББП. Что можно сказать об их отношении?}

Может получиться ББП ($\{n^2\}$ и $\{{n^{-1}}\}$), БМП ($\{n^2\}$ и $\{n\}$). Докажем, что других вариантов нет. Возьмем N1-ый член для некоторого С>1 в ББП и N2  для  в БМП. Тогда отношение любых 2 членов после них будет больше C2 или меньше . Т.к. C2>C, , то выполняются условия на ББП или БМП. 

\paragraph{26.10а.}
\textit{Докажите, что $\frac{n^5 + 3}{n^{10}}$ БМП.}

$\frac{n^5+3}{n^10} = \frac{n^5}{n^10} + \frac{3}{n^10} = \frac{1}{n^5} + \frac{3}{n^10}$. Сумма 2 БМП — БМП.

\paragraph{26.10б.}
\textit{Докажите, что $\frac{3n^6 + 2n^4 - n}{n^9 + 7n^5 - 5n^2 - 2}$ БМП.}

$\frac{3n^6 + 2n^4 - n}{n^9 + 7n^5 - 5n^2 - 2} = \frac{3n^6}{n^9+7n^5-5n^2-2} + \frac{2n^4}{n^9+7n^5-5n^2-2} - \frac{n}{n^9+7n^5-5n^2-2} < \frac{3n^6}{n^9} + \frac{2n^4}{n^9} - \frac{n}{n^9} = \frac{3}{n^3} + \frac{2}{n^5} + \frac{-1}{n^8}$ Сумма 3 БМП — БМП.

\paragraph{26.10в.}
\textit{Докажите, что $\sqrt{\frac{|\sin 3n + \cos 7n|}{2n^2 + 3n}}$ БМП.}

$\sqrt{\frac{|sin 3n + cos 7n|}{2n^2+3n}} < \sqrt{\frac{2}{2n^2+3n}} < \sqrt{\frac{1}{n^2}} = \frac{1}{n}$. БМП.

\paragraph{26.10г.}
\textit{Докажите, что $\frac{3^n + 4^n}{2^n + 5^n}$ БМП.}

$\frac{3^n+4^n}{2^n+5^n} = \frac{3^n}{2^n+5^n} + \frac{4^n}{2^n + 5^n} < \frac{3^n}{5^n} + \frac{4^n}{5^n}$. Сумма 2 БМП — БМП.

\paragraph{26.11.}
\textit{Любую ли последовательность можно представить в виде отношения двух БМП?}

Будем делать две последовательности (делимую и делящую), исходя из результата. В качестве базы возьмем последовательность $\frac{1}{n}$. Пусть n-ый член равен k. Если k > 1, то разделим n-ый член делящей на k. Если k < 1, доммножим n-ый член делимой на k. В обоих случаях делимая и делящая последовательности получаются меньше $\frac{1}{n}$, т.е. тоже БМП.

\paragraph{26.12.}
\textit{По последовательности $\{\alpha_n\}$ построили последовательность $\{\beta_n\}$ так, что $\beta_n = \alpha_{n + 1} - \frac{a_n}{2}$ при любом натуральном n. Докажите, что если последовательность $\{\beta_n\}$ оказалась БМП, то и последовательность $\{\alpha_n\}$ также является БМП.}

По свойству $\beta$ $\alpha_{n+1} - \frac{\alpha{n}}{2} < \varepsilon \rightarrow \alpha_{n+1} < \varepsilon + \frac{\alpha{n}}{2}$. Пусть $\alpha_i$ - последнее число, большее или равное $\varepsilon$. Тогда аналогичными преобразованиями получим $\alpha_n < \varepsilon + \frac{\varepsilon}{2} + \frac{\varepsilon}{4} + ... + \frac{\alpha_i}{2^{n-i}}$. Тогда при некотором n и всех последующих $\frac{\alpha_i}{2^{n-i}} < \varepsilon$. Тогда $a_n < 3 \varepsilon$. Т.к. такое верно для каждого $\varepsilon$, $\alpha$ - БМП.

\medskip\hrule\medskip
\section*{27. Действительные числа}

\paragraph{27.1.}
\textit{Докажите, что множество $\mathbb{R}$ несчетно.}

Как известно, множество всех бесконечных последовательностей из 0 и 1 несчетно. Тогда, т.к. это подмножество БДД, несчетно.

\paragraph{27.2а.}
\textit{Не употребляя отрицаний, дайте определение числа, не являющегося нижней гранью данного множества $M \subset \mathbb{R}$.}

$\exists x \in M$ $b > x$

\paragraph{27.2б.}
\textit{Не употребляя отрицаний, дайте определение множества, неограниченного сверху.}

$\forall b$ $\exists x: b > x$

\paragraph{27.2в.}
\textit{Не употребляя отрицаний, дайте определение неограниченного множества.}

$\forall k \in \mathbb{R} \text{ } \exists n \in A \text{ } |n| > k$

\paragraph{27.3а.}
\textit{Дайте определение минимального элемента.}

$\forall x \in M$ $\exists m \in M$ $m \le$

\paragraph{27.3б.}
\textit{Дайте определение ТВГ.}

$\forall x \in M$ $x \le b$ и $\forall y \in \mathbb{R}$ $b \le y$ или $\exists x \in M$ $x > y$

\paragraph{27.4.}
\textit{Каждое ли ограниченное сверху подмножество в $\mathbb{R}$ имеет максимальный элемент?}

Нет, пример: $\{\frac{-1}{n}\}$.

\paragraph{27.5а.}
\textit{Вычислите ТВГ множества $\{0,3; 0,33; 0,333; 0,3333; 0,33333; . . .\}$.}

ТВГ = $\frac{1}{3}$. Пусть ТВГ меньше — тогда какой-то знак после запятой отличается от 3 в меньшую сторону, а занчит ТВГ меньше одного из элементов. Пусть ТВГ больше — тогда это не ТВГ, т.к. $\frac{1}{3}$ тоже верхняя грань.

\paragraph{27.6а.}
\textit{Верно ли, что любая БДД $b \in \mathbb{R}$ есть ТВГ множества всех БДД, меньших $b$?}

Да, верно, т.к. если ТВГ меньше, то она равна одному из элементов, если больше, то есть меньшая верхняя грань - b. 

\paragraph{27.6б.}
\textit{Верно ли, что любая БДД $b \in \mathbb{R}$ есть ТВГ множества всех конечных БДД, меньших $b$?}

Да, верно, т.к если ТВГ меньше, то она в каком-то знаке отличается от b, значит, если ТВГ  обрезать по этому знаку, то результат будет меньшая конечная БДД, если ТВГ больше, то есть меньшая верхняя грань - b.

\paragraph{27.7а.}
Т.к. b — ТВГ, то меньших верхних граней нет и для любого меньшего числа найдется член множества больше него. Это и есть вопрос задачи.

\paragraph{27.7б.}
Если есть меньшая общая грань, то разница между ней и заданной будет положительной. Тогда, т.к. меньшая верхняя грань — верхняя грань, то больше нее элементов нет, значит достаточно.

\paragraph{27.8.}
\textit{Докажите, что каждое ограниченное сверху подмножество в R имеет точную верхнюю грань, а каждое ограниченное снизу — точную нижнюю.}

Рассмотрим ограниченные сверху, для ограниченных снизу аналогично. Тогда у таких множеств есть верхние грани. Если у множества есть максимальный элемент, то он и будет ТВГ. Тогда пусть его нет и ТВГ нет. Назовем множество М и множество его верхних граней Х. Тогда отметим какую-то верхнюю грань, являющуюся целой БДД и какое-то целое число, не являющееся верхней гранью. Тогда, очевидно, 2-я БДД меньше 1-й. Разделим расстояние между ними на части размера 0.1 — там меняются цифры. Потом разделим на 10 равных частей ту часть, у которой один конец не является верхней гранью, а другой является. Так будем делать и далее. Почему при дальнейшем строительстве такой БДД получим ТВГ? Пусть есть верхняя грань меньше. Тогда она в каком-то знаке отличается от полученной в меньшую сторону, а значит лежит в то части, у которой обе границы не являются верхними гранями, а значит и она сама ею не является. При этом это, очевидно, верхняя грань, т.к. все числа больше нее — верхние грани по той же причине. 

\paragraph{27.9в.}
\textit{Проверьте, что для любых $a, b, c \in R$ верно, что $a(b + c) = ab + ac$.}

В данных операциях используется только сложение и умножение. Обычно результаты таких преобразований равны. Раз по 27.9б сложение и умножение дают то же, что и раньше, для любых пар чисел, то результаты обычно равных операций будут равными и здесь. Тогда множества результатов совпадают, а значит совпадает и их ТВГ.


\paragraph{27.10.}
\textit{Пусть $b \in R$ есть ТВГ конечных БДД $\beta$ с $\beta^2 < 5$. Докажите, что $b$ cуществует и вычислите $b^2$.}

Cуществование $b$ очевидно по 27.8. Докажем, что квадратом такой ТВГ будет 5. Пусть есть меньшая верхняя грань. Тогда она отличается от нее на какое-то число, пусть $\varepsilon$. Тогда ее квадрат равен 5 - $\varepsilon$. Однако на интервале $(5 - \varepsilon; 5)$ есть квадрат конечной БДД, а значит $5 - \varepsilon$ не ТВГ.

\paragraph{27.11.}
Возьмем конечную БДД. Тогда будем домножим ее на 10 в той степени, сколько у нее знаков после запятой и получим . Т.к. БДД конечная, то и степень получится конечная. Тогда БДД представима в виде целого числа, деленного на 10 в конечной степени, т.е. в виде рационального числа.

\paragraph{27.12.}
\textit{Верно ли, что $\forall \alpha \in \mathbb{R}, \varepsilon > 0$ $\exists r_1, r_2 \in \mathbb{Q}$ $\alpha - \varepsilon < r_1 < \alpha < r_2 < \alpha + \varepsilon$?} 

Возьмем БДД $\alpha - \varepsilon$ и $\alpha$. Рассмотрим, в каком первом знаке они различаются. Тогда, если обрезать $\alpha$ по этому знаку, получим БДД, меньшую $\alpha$ и большую $\alpha - \varepsilon$. Аналогично и с $\alpha$ и $\alpha + \varepsilon$.

\paragraph{27.13.}
\textit{Докажите, что для любого сечения $\mathbb{Q} = A_1 \sqcup B_1$ имеет место равенство $sup A_1 = inf A_2$.}

Пусть sup A1 < inf A2. Тогда найдется элемент между ними, а значит не входящий ни в А1, ни в А2. Этого быть не может. Тогда пусть sup A1 > inf A2. Тогда найдется элемент между ними, значит, он будет входить и в А1, и в А2, чего быть не может по определению сечения.

\paragraph{27.14.}
\textit{Докажите, что для любого сечения $Q = A1 \sqcup A2$ выполняется ровно одна из трёх возможностей: либо в $A_1$ есть максимальный элемент, либо в $A_2$ есть минимальный
элемент, либо действительное число $sup A_1 = inf A_2$ иррационально.}

Пусть не выполняется ни одна. Тогда sup A1 = inf A2 должен быть максимальным элементом А1 и минимальным элементом А2, иначе какие-то элементы этих множеств будут лежать в пересечении. Но оно и само лежит в пересечении, чего также быть не может.
Пусть выполняются 2 из них. Одновременно максимальный и минимальный элементы быть не могут, т.к. они должны быть равны по 27.13. Максимальный и минимальный элементы будут sup A1 и inf A2 соответственно, поэтому иррациональность невозможна. 
\newpage
\paragraph{27.15а.}
\textit{Дайте определение суммы действительных чисел Дедекинда.}

Пусть множество $\mathbb{Q}$ делится на $A_1$ и $A_2$ сечением a и на $B_1$ и $B_2$ сечением b. Тогда сумма 2-х сечений будет сечением, которое делит $\mathbb{Q}$ на множество тех элементов, которые получаемы в виде суммы некоторого элемента $A_1$ и некоторого элемента $B_1$. Теперь о том, почему полученное множество - сечение $\mathbb{Q}$. Пусть не сечение - тогда найдется такой элемент между элементами $C_1$ (результата), который ему не принадлежит. Тогда пусть разница такого элемента с большим принадлежащим элементом $\varepsilon$. Она, очевидно, рациональна. Тогда разделим $\varepsilon$ пополам и вычтем из элементов из $A_1$ и $B_1$ половины. Получим члены $A_1$ и $A_2$. Значит, тот самый элемент внутри $C_1$. Значит, это сечение.

\paragraph{27.15б.}
\textit{Дайте определение произведения действительных чисел Дедекинда.}

Пусть множество $\mathbb{Q}$ делится на $A_1$ и $A_2$ сечением a и на $B_1$ и $B_2$ сечением b. Тогда произведение 2-х сечений будет сечением, которое делит $\mathbb{Q}$ на множество тех элементов, которые получаемы в виде произведения некоторого элемента $A_1$ и некоторого элемента $B_1$. Теперь о том, почему полученное множество - сечение $\mathbb{Q}$. Пусть не сечение -тогда найдется такой элемент между элементами $C_1$ (результата), который ему не принадлежит. Тогда пусть отношение такого элемента к большему принадлежащиму элементу $\varepsilon$. Оно, очевидно рациональное. Если оно к тому же и положительно, то вместо половины берем квадратный корень, за исключением этого все так же, как и в 27.15а. Если же отношение отрицательно, то возьмем два положительных элемента, составляющих больший элемент. Домножим один из них на -1, а другой умножим на модуль отношения. Получим тот самый элемент, значит, это сечение.

\paragraph{27.16.}
\textit{Установите сохраняющую арифметические операции биекцию (изоморфизм) между множеством дедекиндовых действительных чисел и множеством действительных чисел? определённых посредством БДД.}

Пусть есть класс БДД. Докажем, что каждому классу БДД соответствует только одно сечение. Каждому БДД пусть соответствует сечение на множество меньших и больших рациональных БДД. Тогда если есть другое сечение, то найдется число, которое будет одновременно больше и меньше БДД, т.е. оно будет равно этой БДД, т.е. получим одно и то же сечение. При этом не найдется такого сечения, которому не соответствует ни одна БДД, и не найдется сечения, которому соответствует 2 БДД, в силу того, что рациональные числа есть на любом интервале.


\paragraph{27.17а.}
\textit{Исходя только из дедекиндова определения действительных чисел докажите теорему о полноте.}

Рассмотрим ограниченные сверху, для ограниченных снизу аналогично. Тогда у таких множеств есть верхние грани. Пусть ТВГ нет. Выберем 2 сечения - у одного inf $A_1$ будет меньше какого-то элемента множества (назовем его сечениями I типа), у другого больше (II типа). Возьмем среднее арифметическое этих двух сечений - мы умеем складывать, делить пополам отрезки тоже умеем. Рассмотрим это сечение. Тогда оно подпадает под какой-то тип. Берем тот отрезок, у которого концы - inf сечений с разными типами и делим его пополам. Так мы получим ТВГ. Почему это ТВГ? Если есть верхняя грань больше, то когда нибудь ее различие с ТВГ превысит длину самого короткого отрезка  из упомянутых ранее, а значит она попадет в отрезок с концами одного типа, т.к. различие не поместится в отрезок.

\medskip\hrule\medskip


\section*{28. Топология прямой - Начало}

\paragraph{28.1.}
\textit{Имеется последовательность отрезков, каждый из которых содержится в предыдущем. Может ли пересечение всех этих отрезков быть пустым?}

Возьмем ТВГ левых концов и ТНГ правых. Пусть ТНГ правых меньше ТВГ левых. Тогда есть отрезок с левым концом больше ТНГ правых концов (пусть расстояние до него $\varepsilon$). Если его длина будет $\frac{\varepsilon}{2}$, то его правый конец будет меньше ТНГ правых, т.е. это не ТНГ. Тогда ТВГ левых будет всегда меньше или равно ТНГ правых. Отрезок с концами в них и будет пересечением, и очевидно, что даже при совпадении концов в перечении будет хотя бы она точка.

\paragraph{28.2.}
\textit{Изменится ли ответ 28.1, если отрезки заменить интервалами?}

Да, изменится. Пусть есть последовательность интервалов такая:
\[(\frac{1}{2}; 1), (\frac{3}{4}; 1), (\frac{7}{8}; 1), ...\] Тогда ТВГ ее левых концов - 1, которое не лежит в этих интервалах, значит, пересечение пусто.

\paragraph{28.3а.}
\textit{Группа естествоиспытателей в течении 6 часов наблюдала за (неравномерно)
ползущей улиткой так, что она всё это время была под присмотром. Каждый наблюдатель
следил за улиткой ровно 1 час без перерывов и зафиксировал, что она проползла за этот
час ровно 1 метр. Могла ли улитка за время всего эксперимента проползти 5м?}

Могла. Есть минимум 5 временных точек, в которых одновременно наблюдают 2 наблюдателя. Пусть в них улитка телепортируется на 1 м. Тогда каждый наблюдатель видел ее перемещение на 1 м, притом их могло быть на 1 меньше, чем наблюдателей, которых было минимум 6.

\paragraph{28.3б.}
\textit{Группа естествоиспытателей в течении 6 часов наблюдала за (неравномерно)
ползущей улиткой так, что она всё это время была под присмотром. Каждый наблюдатель
следил за улиткой ровно 1 час без перерывов и зафиксировал, что она проползла за этот
час ровно 1 метр. Могла ли улитка за время всего эксперимента проползти 10м?}

Очевидно, что могут быть 5 наблюдателей, время наблюдения которыми улитки не пересекалось. Тогда в случае, когда последний из выбранных не донаблюдал до конца эксперимента, можно выбрать еще 5 непересекающихся, каждый из которых будет иметь начало наблюдения во аремя наблюдения соответствующего из первой группы. Тогда наблюдателей 10. Могла быть ситуация, что у каждого наблюдателя было время, когда он сидит один. Тогда, если улитка телепортировалась в это время, то всего она проползла 10 м.


\paragraph{28.3в.}
\textit{Группа естествоиспытателей в течении 6 часов наблюдала за (неравномерно)
ползущей улиткой так, что она всё это время была под присмотром. Каждый наблюдатель
следил за улиткой ровно 1 час без перерывов и зафиксировал, что она проползла за этот
час ровно 1 метр. Могла ли улитка за время всего эксперимента проползти 12м?}

Нет, такого быть не могло. Пусть могло. Тогда перемещение улитки было увидено 12 или более раз. Докажем, что как бы мы не располагали 12 точек на отрезке длины 6, найдется отрезок, содержащий 2 из них. Почему? Пусть можно. Отрежем с концов по отрезку длины 1 - там будет не более 2 точек. Останется отрезок длины 4 с 10 точками. Уберем 2 точки - предполагая, что отрезки, их содержащие, очень мало выступают на остаточный отрезок (чем больше выступают, тем больше средняя плотность точек, тем более невозможно утверждение). Получим отрезок длины 4 с 8 точками. Сделаем аналогичное преобразование, получим отрезок длины 2 с 4 точками. Этот отрезок легко разбить на 2 отрезка с 1 точкой в каждом. Значит 2 точки нигде не лежат, чего быть не может - в них за улиткой, по условию кто-то наблюдал. 

\paragraph{28.4.}
\textit{Докажите, что в любом покрытии отрезка
интервалами найдётся конечный набор интервалов, покрывающий весь отрезок.}

Пусть нельзя. Тогда хотя бы одну половину отрезка нельзя. Будем продолжать так счетное число раз, пока не останется точка (по 28.1., исходя из решения такое возможно). Точка очевидно покрыта некоторым интервалом. Пусть расстояние от точки до ближайшего конца $\varepsilon$. Очевидно, что найдется отрезок, содержащий эту точку, с длиной меньше $\frac{\varepsilon}{2}$. Тогда он покрывается интервалом. Противоречие.

\paragraph{28.5.}
\textit{Из любого ли покрытия отрезка интервалами можно удалить часть так, чтобы оставшиеся интервалы тоже составляли покрытие, но каждую точку накрывало бы не более двух из них?}

Сначала уберем вложенные интервалы и оставим из каждой группы вложенных наибольший. Так количество покрытий некоторых точек уменьшится. Теперь возьмем точку, покрываемую тремя или более интервалами. Возьмем эти три интервала и будем левые концы обозначать за А, правые за В. Сама же точка Т. Занумеруем интервалы по порядку их начал. Конец первого лежит до конца второго, конец второго лежит до конца третьего, т.к. иначе какие-то из них будут вложенными. Т.е. получается такая последовательность точек - $A_1, A_2, A_3, T, B_1, B_2, B_3$. Тогда если взять объединение интервалов 1 и 3, то получим интервал, в который вложен второй. Уберем второй и интервалов станет на один меньше, а точка все еще будет покрыта. Будем делать так с самым левым пересечением более чем 2 интервалов, пока его не будет покрывать 2 интервала. После этого рассмотрим следующий интервал. Сделаем так же. Будем делать так на всех пересечениях и в конце получим требуемое покрытие. Почему все точки будут покрыты хотя бы одним интервалом после этих действий? Т.к. мы убираем интервал, вложенный в объединение некоторых других. 

\newpage

\paragraph{28.6.}
\textit{Изменится ли что-нибудь в предыдущих двух задачах, если отрезок заменить
интервалом?}

В 28.4 изменится. Пусть есть интервал (0, 1). Пример подобен примеру из 28.2:
\[(\frac{1}{2}; \frac{7}{8}), (\frac{3}{4}; \frac{15}{16}), (\frac{7}{8}; \frac{31}{32}), (\frac{15}{16}; \frac{63}{64}) , ...\] Очевидно, что конечного покрытия нет. \\
В 28.5. же не изменится - в доказательстве нигде не используются свойства отрезка, отличающиеся от своййств интервалов.

\paragraph{28.7а.}
\textit{Всегда ли из покрытия отрезка конечным множеством содержащихся внутри него отрезков можно выкинуть часть отрезков так, чтобы оставшиеся по-прежнему покрывали исходный отрезок и их суммарная длина не превышала бы его удвоенной длины?}

Будем действовать аналогично 28.5. Эта задача отличается только тем, что в каком-то конечном числе точек может быть 3 отрезка - заканчивающийся в ней, начинающийся в ней и покрывающий (но не более - иначе вложенность). Объединим заканчивающийся и начинающийся и сделаем так везде, где такая ситуация есть. Получим конечное число отрезков и конечное число точек, не влияющих на длину, при этом во всех точках изначального отрезка не более 2 отрезков. 


\paragraph{28.7б.}
\textit{Всегда ли из покрытия отрезка множеством содержащихся внутри него отрезков можно выкинуть часть отрезков так, чтобы оставшиеся по-прежнему покрывали исходный отрезок и их суммарная длина не превышала бы его удвоенной длины?}

После удаления вложенных получаем, что в каждой точке может начинаться 1 отрезок, заканчиваться 1 отрезок и покрывать ее тоже 1 отрезок. Пересечение начинающегося и заканчивающегося - точка, т.е. не влияет на длину. Объединим их. Объединив все такие отрезки, получим ситуацию, подобную 28.5, и много точек. Точек этих будет, очевидно, счетно (можно разбить отрезки на две чередующиеся группы непересекающихся, отрезок больше интервала, по 13.13 отрезков не более чем счетно на прямой, значит на подмножестве прямой их тоже не более, чем счетно), т.е. на длину они все еще не влияют. Зато в каждой точке будет не более 2 отрезков, значит в сумме будет не более 2 длин изначального.

\paragraph{28.8.}
\textit{Убедитесь, что промежутки (а) ($-\infty$, a); (б) (a, $+\infty$); (в) (a, b) открыты.}

Здесь у интервалов есть 2 типа концов - бесконечностный и где-то внутри прямой. У первого типа концов открытость очевидна - если есть внутренняя точка k, то рассмотрим ее расстояние $r$ до другого конца. Тогда есть $\varepsilon$-окрестность с $\varepsilon = \frac{r}{2}$.
У второго типа концов открытость доказывается очевидно - берем расстояние до ближайшего конца k, далее аналогично.

\paragraph{28.9.}
\textit{Можно ли разбить интервал в объединение двух непересекающихся открытых множеств?}

Возьмем какую-то максимальную $\varepsilon$-окрестность. Она является интервалом. Возьмем конец этого интервала, не принадлежащий этому множеству, т.к. $\varepsilon$-окрестность максимальная. Он должен принадлежать другому множеству, а значит его $\varepsilon$-окрестность должна быть в нем. Однако это конец интервала - все точки левее него лежат в нем, а значит в этой точке нет нужной окрестности, значит, второе не открытое. 

\paragraph{28.10.}
\textit{Докажите, что объединение любого набора открытых множеств открыто.}

Возьмем любую точку нового множества - она лежит в каком-то изначальных множеств, а значит там есть какая-то нужная $\varepsilon$-окрестность, а значит она есть и в объединении.\

\paragraph{28.11}
\textit{(а) Докажите, что пересечение конечного набора открытых множеств открыто. (б) Так ли это для пересечений бесконечных наборов?}

Рассмотрим это пересечение. Тогда возьмем минимальную $\varepsilon$-окрестность - она есть в пересечении, т.к. эта окрестность или большая есть в каждом множестве. Если же $\varepsilon$-окрестности нет, то это пересечение пустое, либо одно из множеств не имеет такой окрестности, а значит не открыто.

\newpage

\paragraph{28.12.}
\textit{Докажите, что всякое открытое множество на прямой представляет собой объединение конечного или счётного набора попарно непересекающихся интервалов, в числе которых допускаются и неограниченные интервалы типа ($-\infty$, a), (a, $+\infty$) и ($-\infty$, $+\infty$).}

Пересекающиеся интервалы объединимы. Объединим же $\varepsilon$-окрестности всех точек. Получаем несколько неперескающихся. По 13.13 очевидно, что если есть разбиение этого множества на непересекающиеся интервалы, то их не более, чем счетно.


\paragraph{28.13а.}
\textit{Может ли бесконечное дискретное множество быть ограниченным?}

Может - например, таким будет множество $\{\frac{1}{2^n}\}$. Почему? Потому, что для каждой степени n будет нужная $\varepsilon$-окрестность радиуса $\varepsilon = \frac{1}{2^{n+2}}$.

\paragraph{28.13б.}
\textit{Может ли бесконечное дискретное множество быть несчётным?}

Пусть может. Для каждого элемента множества возьмем нужную $\varepsilon$-окрестность. Если каждую такую окрестность поделить пополам, то получим много непересекающихся интервалов. По 13.13 их не более, чем счетно. Значит, и элементов такого множества не более, чем счетно.

\medskip \hrule \medskip

\section*{29. Топология прямой - Возрождение}


\paragraph{29.1.}
\textit{Постройте бесконечное множество $M \subset \mathbb{R}$, множество предельных точек которого: (а) пусто; (б) состоит из одной точки; (в) состоит из двух точек; (г) совпадает с $\mathbb{Z} \subset  \mathbb{R}$.} \\
1a. -  Множество $\mathbb{N}$. \\
	1б. - Множество $\{\frac{1}{2^n}\}$. Предельная точка - 0, других, очевидно, нет. \\
	1в. - Множество $\{\frac{1}{2^{n-1}}\} \cup  \{2 + \frac{1}{2^{n-1}-1}\}$. Предельные точки 0 и 2 аналогично 29.1б. \\
	1г. -  Множество $\{k + \frac{1}{2^{n-1}}\}$, $k \in \mathbb{Z}$. Предельные точки - k для каждого отдельного k.
	
	\paragraph{29.2.}
\textit{Может ли множество предельных точек быть множеством всех чисел вида $\frac{1}{k}$ с $k \subset N$?}

Пусть может. Тогда для в любой $\varepsilon$-окрестности любого числа $\frac{1}{k}$ есть член множества. При этом в любой $\varepsilon$-окрестности 0 есть число $\frac{1}{k}$. Тогда в любой $2\varepsilon$-окрестности 0 есть член множества. Значит, 0 тоже является предельной точкой, не входя в множество чисел $\frac{1}{k}$.

\paragraph{29.3.}
\textit{Может ли бесконечное ограниченное множество не иметь предельных точек?}

Пусть множество покрывается отрезком длины $a$. В нем не менее, чем счетное количество членов. Поделим этот отрезок пополам. Тогда в какой-то половине будет не менее, чем счетное количество членов. Делая так и дальше, получаем последовательность вложенных отрезков. Есть хотя бы одна точка, принадлежащая всем отрезкам. Теперь выберем такие элементы множества, что для этого подмножества точка будет предельной. Возьмем на каком-то отрезке какой-то член. Пусть расстоние от него до точки $\varepsilon$. Тогда найдется отрезок длины меньше $\varepsilon$, где и возьмем следующий элемент. Так можно будет брать и дальше. Таким образом, для любой $\varepsilon$-окрестности этой точки найдется элемент, в ней лежащий, т.е. точка предельная.
	
\paragraph{29.4.}
\textit{Докажите, что отрезок и прямая — замкнуты, а интервал и луч — нет.}

У любой предельной точки в любой $\varepsilon$-окрестности есть элемент множества. Рассмотрим с этой точки зрения пункты этой задачи. \\
Отрезок - Пусть есть предельная точка, не входящая в него. Тогда расстояние от этой точки до одного из концов равно $\varepsilon$. Тогда в ее окрестности радиуса, меньше этого, точки множества не будет. Противоречие.\\
Прямая - Предельных точек вне прямой нет, т.к. точек вне прямой нет, а любая предельная точка - точка.\\
Интервал - Конец интервала, очевидно, предельная точка, в нем не лежащая. \\
Луч - Аналогично интервалу. 

\paragraph{29.5а.}
\textit{Бывают ли дискретные незамкнутые множества?}

Пример - $\{\frac{1}{2^n}\}$. Единственная предельная точка - 0, не входящая в множество. При этом оно, очевидно, дискретно.

\paragraph{29.5б.}
\textit{Бывают ли бесконечные дискретные компакты?}

Это множество должно быть бесконечным ограниченным. Тогда у него есть предельная точка по 29.3. Изолированная точка не может быть предельной, поэтому предельная точка не входит в множество. Однако множество замкнуто - противоречие.

\paragraph{29.6a.}
\textit{Разбивается ли отрезок в объединение двух непересекающихся непустых замкнутых множеств?}

Разобьем оба множества на максимальные непересекающиеся элементы, т.е. отрезки, интервалы, полуинтервалы и изолированные точки. Это значит, что элементы множества "чередуются" - нет точки, являющейся предельной для двух элементов одного множества одновременно и принадлежащей ему.
Сначала докажем, что изолированных точек ни в одном множестве нет. Пусть она есть в одном из них. Тогда она предельна для другого множества, т.к. все точки ее некоторой окрестности принадлежат другому множеству. Значит, второе множество не замкнуто, противоречие. Также не может быть интервалов и полуинтервалов --- его конец является предельной точкой, не входящей в нужное множество. Значит все элементы --- отрезки. Возьмем первые два. У них есть общий конец. Он должен принадлежать обоим множествам, иначе одно не замкнуто, но такого быть не может по условию. Противоречие.

\paragraph{29.6б.}
\textit{Перечислите все одновременно открытые и замкнутые подмножества в $\mathbb{R}$.}

Открытое множество всегда разбивается на непересекающиеся интервалы (28.12). Во всех них будут предельные точки, не входящие в множество, а значит итоговое множество не будет замкнутым. Такого можно избежать двумя способами - сделать интервалом всю прямую или убрать интервалы вовсе. В первом случае получаем прямую, во втором пустое множество. И то, и то замкнуто. 

\paragraph{29.7.}
\textit{Докажите, что множество $Z \subset \mathbb{R}$ замкнуто тогда и только тогда, когда его дополнение $\mathbb{R} \backslash Z$ открыто.}

Множество, замкнутость которого мы предполагаем, назовем А. Другое же множество - В.

\textit{п.1 - Замкнуто, если дополнение открыто.} Пусть не замкнуто. Тогда есть предельная точка замкнутого, не принадлежащая ему. Тогда она принадлежит В . Тогда, по определению открытого, она должна быть внутренней, т.е. есть ее $\varepsilon$-окрестность полностью лежит в В. Однако она предельная для А, т.е. в любой ее $\varepsilon$-окрестности есть хотя бы одна точка А, чего одновременно быть не может.

\textit{п.2 - Если замкнуто, то дополнение открыто.} Пусть не так. Тогда в В есть не внутренняя точка, т.е. у нее нет окрестности, полностью содержащийся в В. Тогда любая ее $\varepsilon$-окрестность содержит точку из А, т.е. она для него предельная. Однако А замкнуто, т.е. содержит все свои предельные точки, а эта - в В.

\paragraph{29.8.}
\textit{Докажите, что пересечение любого набора замкнутых множеств замкнуто.}

Пусть не так. Тогда у пересечения есть предельная точка, не содержащаяся в нем. Однако она присутствует во всех пересеченных замкнутых множествах либо не является предельной для них. В первом случае она внутри пересечения. Во втором случае любая ее $\varepsilon$-окрестность содержит точки пересечения, а значит всех пересеченных множеств, т.е. она не может перестать быть предельной.

\paragraph{29.9а.}
\textit{Докажите, что объединение конечного набора замкнутых множеств замкнуто.}

Пусть это не так. Тогда его дополнение не открыто. Егои дополнение - пересечение дополнений к замкнутым множествам, т.е. пересечение открытых, являющееся открытым. Значит, по 29.8 объединение замкнутых замкнуто.

\paragraph{29.9б.}
\textit{Так ли это для бесконечных наборов?}

Возьмем такую последовательность отрезков: $a_k = [-1, -1/k]$. Объединим все отрезки. Тогда точка 0 предельная, т.к. для любого $\varepsilon < 0$ найдется такое k, что $-1/k > \varepsilon$. Однако очевидно, что в объединении ее нет. Тогда полученное множество не замкнуто.

\paragraph{29.10.}
\textit{Докажите, что любая последовательность вложенных компактов $K_1 \supset K_2 \supset K_3 \supset ... $ имеет непустое пересечение: $\bigcap K_n \neq \varnothing $.}

Пусть пересечение пусто. Тогда объединение вложенных открытых множеств, являющихся дополнениями компактов --- прямая. Тогда, в силу их вложенности, найдется дополнение, являющееся прямой. Значит, то, что он дополняет, должно являться пустым множеством. Однако это должен быть компакт, а он не пуст. 

\paragraph{29.11.}
\textit{Докажите, что непустое $K \subset \mathbb{R}$ компакт
тогда и только тогда, когда любое его покрытие интервалами содержит конечное подпокрытие.}

\textit{п.1 - Компакт, если любое его покрытие интервалами содержит конечное подпокрытие.} Пусть не так. Тогда в К есть предельная точка не внутри К. В любой ее $\varepsilon$-окрестности есть точка из К. Возьмем какую-то из них и разделим ее на слои со все уменьшающейся толщиной (например, $\frac{1}{2^n}$ от самого внешнего). Каждый слой будет объединением 2 интервалов. Теперь расширим каждый слой на один слой во внутреннюю и наружную стороны. Получим некоторое счетное покрытие этой точки, притом, очевидно, без конечного подпокрытия. Из-за того, что на каждом слое есть точка К глубже, то оно будет счетным и для К, что противоречит условию.

\textit{п.2 - Если компакт, то любое его покрытие интервалами содержит конечное подпокрытие.} По 28.4 любое покрытие интервалами отрезка содержит конечное подпокрытие. Тогда если докажем, что компакт - объединение конечного числа непересекающихся отрезков и изолированных точек, то требуемое утверждение будет верно. Докажем же это. Компакт не имеет предельных точек вне себя. Рассмотрим все предельные точки, не являющиеся внутренними или изолированными. В этих местах и только в этих местах будет "смена" \text{} точек, не принадлежащих компакту, и принадлежащих ему, и наоборот. Почему? Что здесь будет смена очевидно, т.к. иначе точка будет либо внутренней, либо изолированной. Смена будет только здесь, т.к. если она произойдет где-то еще, то эта точка не будет ни изолированной, ни внутренней, однако предельной. Тогда, соблюдая четность, разделяем компакт на отрезки, потом прибавляем к этому изолированные точки. Они, очевидно, каждый в отдельности имеют конечное подпокрытие, а значит, что и все вместе имеют конечное подпокрытие.

\paragraph{29.12.}
\textit{Докажите, что канторово множество имеет мощность континуума, но покрывается конечным набором интервалов со сколь угодно малой суммой длин.}

Очевидно, что длина каждого следующего компакта изменяется на $\frac{2}{3}$. Т.е. длина $K_n$ равна $(\frac{2}{3})^n$. Тогда последовательность длин этих компактов - БМП. Поэтому для любого количества интервалов в покрытии найдется такое n, длина которого будет меньше, чем суммарная длина интервалов, поделенная на их количество, значит, конечное покрытие найдется. Мощность континнума оно имеет, т.к. это равномощно множеству бесконечных последовательностей из 0 и 1, которое континуум.

\paragraph{29.13а.}
\textit{Содержит ли канторово множество интервалы?}

Пусть содержит. Тогда для любой длины интервала найдется компакт, образующий канторово множество, с суммарной длиной меньше него. Значит будут точки интервала, не лежащие в компакте, а значит и в ео пересечении с чем угодно, т.е. в канторов множестве.

\paragraph{29.13б.}
\textit{Содержит ли канторово множество изолированные точки?}

Пусть содержит. Количество отрезков в $K_n$ - $2^n$. Все они, очевидно, равной длины. У изолированной точки с обеих сторон - пустота. Значит, она должна занимать место отрезка. Отрезков - счетное число, значит и изолированных точек будет счетное число. Тогда канторово множество будет счетно, что противоречит 29.12.

\medskip \hrule \medskip

\section*{30. Пределы.}

\paragraph{30.1.}

\textit{Приведите пример последовательности, которая имеет (а) ноль; (б) одну; (в) две; (г) $N$, для некоторого фиксированного $N \subset \mathbb{N}$; (д) счётное количество предельных точек.} \\
1а. $a_n = n$ \\
1б. $a_n = \frac{1}{n}$ \\
1г. $a_n = k + \frac{1}{n}$, где $k \equiv n \pmod{N}$ \\
1д. (Эту последовательность сложно задать нормально). Сделаем таблицу, в ячейках которой расположены числа вида $k + \frac{1}{n}$, где k - номер строки, n - столбца. Тогда таблица имеет известный верхний левый угол. Начиная с него, будем добавлять числа в последовательность диагоналями, параллельными побочной.

\newpage

\paragraph{30.2.}
\textit{Докажите, что $A = \lim\limits_{n \to \infty} a_ n$ тогда и только тогда, когда существует такая бесконечно малая $\{\alpha_n\}$, что $a_n = A + \alpha_n$.}

\textit{п. 1 - Если  $A = \lim\limits_{n \to \infty} a_ n$, то есть БМП.} 
Возьмем эту последовательность и сдвинем так, чтобы она сходилась к 0 и расстояния между членами ее не менялось. Она останется сходящейся к 0, т.к. $\varepsilon$-окрестности не изменились. При этом каждый член уменьшился на А. Эта последовательность стала БМП, т.к. оно эквивалентно сходимости к 0.

\textit{п. 2 - Если есть БМП, то $A = \lim\limits_{n \to \infty} a_ n$ }. 
Совершим сдвиг в другую сторону - $\varepsilon$-окрестности не изменились, поэтому она будет сходиться к А. 

\paragraph{30.4.}
\textit{Докажите, что если $A = \lim\limits_{n \to \infty} a_ n$, то у множества $\{a_n\}$ имеет ровно одна предельная точка.}

Из определения предела очевидно, что он является предельной точкой, поэтому их не менее одной. Пусть есть две. Тогда есть две непересекающиеся окрестности этих точек, в одной точки все, кроме конечного числа. Значит, во второй не более конечного числа. Значит, из них можно отобрать ближайшую к предельной точке. Тогда можно сделать ее окрестность радиусом в половину расстояния от точки до предельной точки. В этой окрестности членов последовательности нет, поэтому она не предельная.

\paragraph{30.5.}
\textit{Докажите, что $A$ --- предельная точка последовательности $\{a_n\}$ тогда и только тогда, когда существует подпоследовательность $a_{n_k} \to A$ при $n_k \to +\infty$.}

\textit{п.1 - Если предельная точка, то есть подпоследовательность.}
Берем любую $\varepsilon$-окрестность точки. В ней есть член последовательности. Берем вложенную окрестность такую, что выбранная точка в ней не лежит, уменьшая радиусы окрестности в 2 раза. При этом будем выбирать следующую точку так, чтобы она была по номеру в изначальной последовательности больше, чем предыдущие. Почему так можно сделать? По определению предельной точки в любой ее окрестности лежит бесконечное количество точек последовательности, при этом точек с неподходящих номером меньше - конечное. По определению предельной точки так выбирать точки можно счетное количество раз. Значит, эти точки образуют подпоследовательность, сходящуюся к $A$.

\textit{п.2 - Если подпоследовательность, то предел - предельная точка.}
Если подпоследовательность сходится к $A$, то в любой окрестности $A$ есть хотя бы одна точка подпоследовательности, значит она предельная для подпоследовательности.

\paragraph{30.6.}
\textit{Докажите, что $a: N \to \mathbb{R}$ сходится если и только если является фундаментальной.}

\textit{п.1 - Сходится, если фундаментальна.}
Возьмем $\varepsilon$. Тогда найдется такое N, что все члены после него будут лежать в его некотором отрезке. Т.е. на этом отрезке будет бесконечное количество точек. Тогда на нем для любого $\varepsilon$ найдется отрезок, на котором будет лежать бесконечное множество точек, притом если $\varepsilon_1 > \varepsilon_2$, то в первом найдется соответствующий отрезок, соответствующий $\varepsilon_2$.  Тогда можно уменьшать $\varepsilon$, например, деля получаемый отрезок пополам и взимая часть с бесконечным количеством в ней, счетное количество раз. Получим предельную точку подпоследовательности, по 30.4. исходная сходится.

\textit{п.2 - Фундаментальная, если сходится.}
Возьмем $\varepsilon$-окрестность точки, к которой сходится последовательность. В ней все, кроме конечного числа, члены последовательности, т.е. среди отсутствующих найдется последний. Начиная со следующего, все лежат в ней, а значит расстояние между любыми следующими не более $2\varepsilon$.  Значит, последовательность фундаментальна, т.к. так можно делать с любым $\varepsilon$.



\paragraph{30.8}
\textit{Докажите, что в $\mathbb{R}$ сходится любая ограниченная монотонная последовательность.}

Возьмем отрезок между ТВГ и ТНГ. В нем все точки. Поделим пополам. Т.к. последовательность монотонна, то в половине с ТВГ точек бесконечно много. Поделим получившийся отрезок еще счетное количество раз, получим предельность ТВГ, а значит, что последовательность сходится.


\paragraph{30.9а}
\textit{Докажите, что всякая сходящаяся ограничена.}

Возьмем какую-нибудь окрестность предела. Тогда в ней лежит все, кроме конечного числа, точки. Множество их ограничено. Всех остальных точек конечно, значит, их множество ограничено. Значит, ограничено и объединение.

\newpage

\paragraph{30.9б.}
\textit{Приведите пример ограниченной расходящейся последовательности.}

Будем строить последовательность из таких кусков - $\{1 - tfrac{1}{n}; -1 + \tfrac{1}{n}\}$. Тогда она состоит из двух монотонных - убывающей и возрастающей, притом члены убывающей всегда меньше членов возрастающей. Тогда расстояние между членами всегда увеличивается, последовательность не функциональная, а потому расходящаяся.

\paragraph{30.10б.}
\textit{Пусть $A$ — предел последовательности $\{a_n\}$. Докажите, что если
в $\{a_n\}$ бесконечно много положительных и отрицательных членов, то $A$ = 0.}

Пусть это не так и $A \neq 0$. Тогда найдется окрестность А, в которой нет 0. Тогда все числа в ней будут олного знака. По определению предела, за пределами этой окрестности будет конечное число членов. Значит, членов одного знака будет конечное число, что противоречит условию.


\paragraph{30.11a.}
\textit{Последовательности $\{a_n\}$ и $\{b_n\}$ имеют пределы A и B соответственно. Докажите, что тогда $\underset{n \to +\infty}{\lim} (a_n \pm b_n) = A \pm B$.}

Пусть $\{a_n\} = A + \alpha_n$; $\{b_n\} = B + \beta_n$, где $\alpha_n, \beta_n$ --- БМП. Тогда $\underset{n \to +\infty}{\lim} (a_n \pm b_n) = \underset{n \to +\infty}{\lim} (A + B + \alpha_n + \beta_n)$. Тогда по 30.2 пределом будет $A + B$, т.к. $\alpha_n + \beta_n$ --- БМП.

\paragraph{30.11б.}
\textit{Последовательности $\{a_n\}$ и $\{b_n\}$ имеют пределы A и B соответственно. Докажите, что тогда  $\underset{n \to +\infty}{\lim} (a_n b_n) = AB$.}

Пусть $\{a_n\} = A + \alpha_n$; $\{b_n\} = B + \beta_n$, где $\alpha_n, \beta_n$ --- БМП. Тогда $\underset{n \to +\infty}{\lim} (a_n \pm b_n) = \underset{n \to +\infty}{\lim} (AB + B\alpha_n + A\beta_n + \alpha_n\beta_n)$. Слагаемые 2 и 3 этой скобки - БМП, т.к. это БМП, умноженные на константу, 4 - БМП, как произведение БМП. Тогда по 30.2 предел $AB$.


\paragraph{30.11в.}
\textit{Последовательности $\{a_n\}$ и $\{b_n\}$ имеют пределы A и B соответственно. Докажите, что тогда если $B \neq 0$ и все элементы последовательности $\{b_n\}$ отличны от нуля, то $\underset{n \to +\infty}{\lim} (\tfrac{a_n}{ b_n}) = \tfrac{A}{B}$.}

Пусть $\{a_n\} = A + \alpha_n$; $\{b_n\} = B + \beta_n$, где $\alpha_n, \beta_n$ --- БМП. Тогда $\underset{n \to +\infty}{\lim} (\tfrac{a_n}{ b_n}) = \tfrac{A + \alpha_n}{B + \beta_n}= \underset{n \to +\infty}{\lim} (\tfrac{A}{B} + \tfrac{A + \alpha_n}{B + \beta_n} - \tfrac{A}{B} = \underset{n \to +\infty}{\lim} (\tfrac{A}{B} + \tfrac{B(A + \alpha_n) - A(B + \beta_n)}{B(B + \beta_n)}= \underset{n \to +\infty}{\lim} (\tfrac{A}{B} + \tfrac{B\alpha_n - A\beta_n}{B(B + \beta_n)})$. 2-е слагаемое - БМП, деленное на константу + БМП, что меньше, чем БМП, деленное на константу, т.е. БМП. Тогда по 30.2.

\paragraph{30.12а.}
\textit{Пусть последовательности  $\{a_n\}$ и $\{b_n\}$ сходятся. Докажите, что если почти для всех $n \in \mathbb{N}$ выполняется условие $a_n = b_n$, то $\underset{n \to +\infty}{\lim} a_n =\underset{n \to +\infty}{\lim} b_n$.}

Тогда конечное число членов не равны между собой. Тогда среди них есть последний. Если есть последний член, значит все после него равны. Значит, найдется такая окрестность пределов $A$ и $B$, что в них все члены последовательностей будут равными. Значит, и пределы будут равными, т.к. если это не так, то найдется пара непересекающихся окрестностей с равными членами, что невозможно.

\paragraph{30.12б.}
\textit{Пусть последовательности  $\{a_n\}$ и $\{b_n\}$ сходятся. Докажите, что если почти для всех $n \in \mathbb{N}$ выполняется условие $a_n \geq b_n$, то $\underset{n \to +\infty}{\lim} a_n \geq \underset{n \to +\infty}{\lim} b_n$.}

Тогда найдется пара непересекающихся окрестностей пределов, в которых $a_n \geq b_n$. Тогда пределы очевидно не равны, при этом если $A < B$, то и все точки окрестности $A$ меньше любой точки окрестности $B$, что неверно по условию.

\paragraph{30.12в.}
\textit{Останется ли верным последнее утверждение, если в нем все знаки нестрогого неравенства заменить на знаки строгого неравенства?}

Нет, не останется. Пример --- последовательности $\{\frac{1}{n}\}$ и $\{-\frac{1}{n}\}$. У обеих предел 0, при этом одна больше другой.

\paragraph{30.13.}
\textit{Пусть последовательности $\{a_n\}$,  $\{b_n\}$, $\{c_n\}$ таковы, что почти для почти для всех $n \in \mathbb{N}$ выполнено неравенства $a_n < c_n < b_n$ и $\underset{n \to \infty}{\lim} a_n = \underset{n \to \infty}{\lim} b_n  = A$. Докажите, что тогда $\underset{n \to \infty}{\lim} c_n  = A$.}

Для любого $\varepsilon$ найдется меньшая окрестность А, в которой есть и $a_n$, и $b_n$. Значит, есть и $c_n$. При этом очевидно, что такая окрестность найдется для всех $n$, при которых верно $a_n < c_n < b_n$, то есть почти все. Тогда, взяв любую окрестность А, найдем в ней почти все точки $c_n$. Значит, А - предел $c_n$.

\paragraph{30.14а.}
\textit{Алиса записала определение последовательности, имеющей предел, следующим образом: \\ «$\exists N \thickspace \forall \varepsilon > 0 \thickspace \forall n > N \thickspace |a_n - A| < \varepsilon$». Опишите множество последовательностей, которые задает данное определение.}

Расшифруем определение. Существует $N$ такое, что для любого $\varepsilon > 0$ и $n > N \thickspace |a_n - A| < \varepsilon$. Тогда любые точки этой последовательности от предела удалены на 0, иначе найдется $\varepsilon$ такой, что $ |a_n - A| > \varepsilon$. То есть это последовательности, состоящие из равных членов.


\paragraph{30.14б.}
\textit{Боб записал определение последовательности, имеющей предел, следующим образом: \\ «$\exists \varepsilon > 0 \thickspace \forall N \thickspace \forall n > N \thickspace |a_n - A| < \varepsilon$». Опишите множество последовательностей, которые задает данное определение.}

Это расшифровывается так: Существует $\varepsilon > 0$ такой, что для любого $N$ и $n>N$  $  |a_n - A| < \varepsilon$. Т.е. для любого $n$ это будет верно. Т.е. все элементы лежат в $\varepsilon$-окрестности А, т.е. последовательность ограниченная.

\end{document}
